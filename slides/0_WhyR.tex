% Created 2017-10-23 Mon 23:36
% Intended LaTeX compiler: pdflatex
\documentclass[11pt,xcolor=dvipsnames]{beamer}
\usepackage[utf8]{inputenc}
\usepackage[T1]{fontenc}
\usepackage{graphicx}
\usepackage{grffile}
\usepackage{longtable}
\usepackage{wrapfig}
\usepackage{rotating}
\usepackage[normalem]{ulem}
\usepackage{amsmath}
\usepackage{textcomp}
\usepackage{amssymb}
\usepackage{capt-of}
\usepackage{hyperref}
\newcommand{\bottomcite}[1]{\fbox{\vbox{\footnotesize #1}}}
\usepackage{multicol}
\input{org-babel.tex}
\usetheme{default}
\author{schnorr}
\date{\today}
\title{Why R? \linebreak (CMP595 PPGC/INF/UFRGS)}
\hypersetup{
 pdfauthor={schnorr},
 pdftitle={Motivation for a Rigourous Analysis \linebreak (CMP595 PPGC/INF/UFRGS)},
 pdfkeywords={},
 pdfsubject={},
 pdfcreator={Emacs 25.2.2 (Org mode 9.0.1)}, 
 pdflang={English}}
\begin{document}

{\setbeamertemplate{footline}{} 

\author{Lucas Mello Schnorr, Jean-Marc Vincent}

\date{INF/UFRGS \newline Porto Alegre, Brazil -- October 2018}

\titlegraphic{
    \includegraphics[scale=1.4]{./logo/ufrgs2.png}
    \hspace{1cm}
    \includegraphics[scale=1]{./logo/licia-small.png}
    \hspace{1cm}
    \includegraphics[scale=0.3]{./logo/uga.png}
}
\maketitle
}

\section{Main}


\begin{frame}[label=sec-4-1-1]{Why R?}
R is a great language for data analysis and statistics
\begin{itemize}
\item Open-source and multi-platform
\item Very expressive with high-level constructs
\item Excellent graphics
\item Widely used in academia and business
\item Very active community
\begin{itemize}
\item Documentation, FAQ on \url{http://stackoverflow.com/questions/tagged/r}
\end{itemize}
\item Great integration with other tools
\end{itemize}
\end{frame}
\begin{frame}[label=sec-4-1-2]{Why is such R a pain for computer scientists?}
\begin{itemize}
\item R is \alert{not} really a \alert{programming} language
\item Documentation is for statisticians
\item Default plots are \textit{cumbersome} (meaningful)
\item Summaries are \textit{cryptic} (precise)
\item \alert{Steep learning curve} even for us, computer scientists whereas we
generally switch seamlessly from a language to another!  That's
frustrating! ;)
\end{itemize}
\end{frame}
\begin{frame}[label=sec-4-1-3]{Do's and dont's}
\textit{R is high level, I'll do everything myself}
\begin{itemize}
\item CTAN comprises 4,334 \TeX{}, \LaTeX{}, and related packages and
tools. Most of you do not use plain \TeX{}.
\item Currently, the CRAN package repository features 4,030 available
packages.
\item How do you know which one to use??? Many of them are highly
exotic (not to say useless to you).
\begin{center}
I learnt with \url{http://www.r-bloggers.com/}
\end{center}
\end{itemize}


\begin{itemize}
\item Lots of introductions but not necessarily what you're looking
for so \alert{I'll give you a short tour}. 

You should quickly realize though that you need proper training
in statistics and data analysis if you do not want tell
nonsense.

\item Again, you should read \alert{Jain's book on The Art of Computer Systems
Performance Analysis}

\item You may want to \alert{follow online courses}:
\begin{itemize}
\item \url{https://www.coursera.org/course/compdata}
\item \url{https://www.coursera.org/course/repdata}
\end{itemize}
\end{itemize}
\end{frame}
\begin{frame}[fragile,label=sec-4-1-4]{Install and run R on debian}
 \small
\begin{verbatim}
apt-cache search r
\end{verbatim}
Err, that's not very useful :) It's the same when searching on
google but once the filter bubble is set up, it gets better\ldots{}
\begin{verbatim}
sudo apt-get install r-base
\end{verbatim}

\begin{verbatim}
R
\end{verbatim}

\scriptsize
\begin{verbatim}
R version 3.2.0 (2015-04-16) -- "Full of Ingredients"
Copyright (C) 2015 The R Foundation for Statistical Computing
Platform: x86_64-pc-linux-gnu (64-bit)

R is free software and comes with ABSOLUTELY NO WARRANTY.
You are welcome to redistribute it under certain conditions.
Type 'license()' or 'licence()' for distribution details.

R is a collaborative project with many contributors.
Type 'contributors()' for more information and
'citation()' on how to cite R or R packages in publications.

Type 'demo()' for some demos, 'help()' for on-line help, or
'help.start()' for an HTML browser interface to help.
Type 'q()' to quit R.

>
\end{verbatim}
\end{frame}

\begin{frame}[fragile,label=sec-4-1-5]{Install a few cool packages}
 R has it's own package management mechanism so just run R and type the
following commands:
\begin{itemize}
\item \texttt{ddply}, \texttt{reshape} and \texttt{ggplot2} by Hadley Wickham (\url{http://had.co.nz/})
\begin{verbatim}
install.packages("plyr")    
  # or better: install.packages("dplyr")
install.packages("reshape") 
  # or better; install.packages("tidyr")
install.packages("ggplot2")
\end{verbatim}
\item \texttt{knitR} by (Yihui Xie) \url{http://yihui.name/knitr/}
\begin{verbatim}
install.packages("knitr")
\end{verbatim}
\end{itemize}
\end{frame}
\begin{frame}[fragile,label=sec-4-1-6]{IDE}
 Using R interactively is nice but quickly becomes painful so at some
point, you'll want an IDE.

\medskip

Emacs is great but you'll need \emph{Emacs Speaks Statistics}
\begin{verbatim}
sudo apt-get install ess
\end{verbatim}
\medskip

\begin{center}
In this tutorial, I will briefly show you \alert{rstudio}
(\url{https://www.rstudio.com/}) and later how to use \texttt{org-mode}
\end{center}
\end{frame}
\subsection{Reproducible Documents: knitR}
\label{sec-4-2}
\begin{frame}[label=sec-4-2-1]{Rstudio screenshot}
\vspace{-.5cm}
\begin{center}
  \includegraphics[height=9cm]{./img/rstudio_shot.png}
\end{center}
\end{frame}
\begin{frame}[fragile,label=sec-4-2-2]{Reproducible analysis in Markdown + R}
 \begin{itemize}
\item Create a new \alert{R Markdown} document (Rmd) in rstudio
\item R chunks are interspersed with \texttt{```\{r\}} and \texttt{```}
\item Inline R code: \texttt{`r sin(2+2)`}
\item You can \alert{knit} the document and share it via \alert{rpubs}
\item R chunks can be sent to the top-level with \texttt{Alt-Ctrl-c}
\item I usually work mostly with the current environment and only knit in
the end
\item Other engines can be used (use rstudio \alert{completion})
\begin{verbatim}
```{r engine='sh'}
ls /tmp/
```
\end{verbatim}
\item Makes \alert{reproducible analysis as simple as one click}
\item Great tool for quick analysis for self and colleagues, homeworks, \ldots{}
\end{itemize}
\end{frame}


%%%%%%%%%%%%%%%%%%%%%%%%%%%%%%%%%%%%%%%%%%%%%%%%%%%%%%
%%%%%%%%%%%%%%%%%%%%%%%%%%%%%%%%%%%%%%%%%%%%%%%%%%%%%%
\end{document}
