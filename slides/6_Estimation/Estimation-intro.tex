%%%%%%%%%%%%%%%%%%%%%%%%%%%%%%%%%%%%%%%%%%%%%%%%%%%%%%%%%%%%
%%  This Beamer template was created by Cameron Bracken.
%%  Anyone can freely use or modify it for any purpose
%%  without attribution.
%%
%%  Last Modified: January 9, 2009
%%

%\documentclass[xcolor=x11names,compress,8pt]{beamer}
\documentclass[xcolor=x11names,compress,8pt,handout]{beamer}

%%% langue
%\usepackage[francais]{babel}

\usepackage[T1]{fontenc}
\usepackage[utf8]{inputenc}

%%% mathématiques
\usepackage{amsmath,amsfonts,amssymb,newlfont,latexsym,fleqn}

 \def\leq{\leqslant}
 \def\geq{\geqslant}
 
 \def\real{\mathbb{R}}
 \def\Prob{\mathbb{P}}
 
 \def\integer{\mathbb{N}}
 \def\relative{\mathbb{Z}}
 \def\Esp{\mathbb{E}}
  \def\Var{\mathbb{V}ar}


%%%%% hyperliens
\usepackage{hyperref,url}
\hypersetup{
dvips,
backref=true, %permet d'ajouter des liens dans...
pagebackref=true,%...les bibliographies
hyperindex=true, %ajoute des liens dans les index.
colorlinks=true, %colorise les liens
breaklinks=true, %permet le retour à la ligne dans les liens trop longs
urlcolor= blue, %couleur des hyperliens
linkcolor= blue, %couleur des liens internes
bookmarks=true, %créé des signets pour Acrobat
bookmarksopen=true} 

%% General document %%%%%%%%%%%%%%%%%%%%%%%%%%%%%%%%%%
\usepackage{graphicx}

\graphicspath{{logos/}{images/}{figures/}{photos/}}

\usepackage{figlatex}%

\usepackage{tikz}
\usepackage{pgfplots}

%\usetikzlibrary{decorations.fractals}
%%%%%%%%%%%%%%%%%%%%%%%%%%%%%%%%%%%%%%%%%%%%%%%%%%%%%%


%% Beamer Layout %%%%%%%%%%%%%%%%%%%%%%%%%%%%%%%%%%
\useoutertheme[subsection=false,shadow]{miniframes}
\useinnertheme{default}
\usefonttheme{serif}
\usepackage{palatino}

\setbeamerfont{title like}{shape=\scshape}
\setbeamerfont{frametitle}{shape=\scshape}

\setbeamercolor*{lower separation line head}{bg=DeepSkyBlue4} 
\setbeamercolor*{normal text}{fg=black,bg=white} 
\setbeamercolor*{alerted text}{fg=red} 
\setbeamercolor*{example text}{fg=black} 
\setbeamercolor*{structure}{fg=black} 
 
\setbeamercolor*{palette tertiary}{fg=black,bg=black!10} 
\setbeamercolor*{palette quaternary}{fg=black,bg=black!10} 

\renewcommand{\(}{\begin{columns}}
\renewcommand{\)}{\end{columns}}
\newcommand{\<}[1]{\begin{column}{#1}}
\renewcommand{\>}{\end{column}}

%%%%%%%%%%%%%%%%%%%%%%%%%%%%%%%%%%%%%%%%%%%%%%%%%%
\usefonttheme{progressbar}
\useoutertheme{progressbar}
\useinnertheme{progressbar}

%%%%%%%%%%%%%%%%%%%%%%%%%%%%%%%%%%%%%%%%%%%%%%%%%%
\usepackage[vlined]{algorithm2e}
\definecolor{gris05}{gray}{0.95}

\SetKw{KwDownTo}{downto}
\SetKwIF{If}{ElseIf}{Else}{if}{}{else if}{else}{endif}
\SetKwFor{For}{for}{}{endfor}
\SetKwFor{While}{while}{}{endw}%

\SetKwInput{KwIn}{Input}%
\SetKwInput{KwOut}{Output}%
\SetKwInput{KwData}{Données}%
\SetKwInput{KwResult}{Résultat}%


\setbeamertemplate{itemize item}[triangle]  
\setbeamertemplate{enumerate item}[diamond] 

\setbeamertemplate{navigation symbols}{}

%\beamertemplatetransparentcovereddynamic

% \setbeamercovered{transparent}

\pgfdeclareimage[height=0.5cm]{university-logo}{logos/logoUGA.jpg}
  
\logo{\pgfuseimage{university-logo}}

\title[Estimation] % (optional, use only with long paper titles)
{Estimation}
\subtitle{How to get information from samples}

\author% (optional, use only with lots of authors)
{		
\textbf{\{Arnaud.Legrand,Jean-Marc.Vincen\}t@imag.fr}
}
% - Give the names in the same order as the appear in the paper.
% - Use the \inst{?} command only if the authors have different
%   affiliation.

\institute[Université de Grenoble-Alpes] % (optional, but mostly needed)
{%
{\large Université  de Grenoble-Alpes, UFR IM$^\text{2}$AG\\
M2R-MOSIG Scientific Methodology}
}
% - Use the \inst command only if there are several affiliations.
% - Keep it simple, no one is interested in your street address.

\date[Grenoble 2016] % (optional, should be abbreviation of conference name)
{\includegraphics[width=4cm]{logos/logoUGA.jpg}\\November 2016
}



 \AtBeginSection[]
{
  \begin{frame}<beamer>
    \frametitle{Codes correcteurs d'erreur }
    \tableofcontents[currentsection]
  \end{frame}
}

\begin{document}

\begin{frame}
\titlepage
\end{frame}

%%%%%%%%%%%%%%%%%%%%%%%%%%%%%%%%%%%%%%%%%%%%%%%%%%%%%%
%%%%%%%%%%%%%%%%%%%%%%%%%%%%%%%%%%%%%%%%%%%%%%%%%%%%%%
\section[{\scshape The problem}]{{\scshape The problem} : Estimate a System Parameter }

\begin{frame}{Protocol Validation}
\begin{block}{Typical random access protocol to a common channel (CSMA family)}

\colorbox{gris05}{
\hspace{0.05\textwidth}\begin{minipage}[t]{0.925\textwidth}
%\LinesNotNumbered
\SetKwFunction{Wait}{Wait}
\SetKwFunction{Collision}{Collision}
\SetKwFunction{Random}{Random}
\SetKwData{Val}{Valeur}

\DontPrintSemicolon
\BlankLine
\hspace{-2.5ex}Protocol-Send ($M$)\;

%\KwData{Configuration de jeu, c'est au joueur A de jouer }
%\KwResult{meilleure valeur  (si B joue parfaitement)}

\While{Message is not sent}{
Send($M$)

\If{\Collision}{
W=\Random ($I_n$)\;

$n=n+1$\;

$I_{n+1}=g(n, I_n)$\;

\Wait(W)
}
}
\end{minipage}
}
\end{block}
\begin{alertblock}<2->{}
What should be the \textit{amount of time} ?
\end{alertblock}
\begin{block}<3->{Protocol dimensioning}
Waiting time :
\begin{itemize}
\item Random
\item Uniformly distributed on an interval $[0,I_n]$
\item Length of the interval depends on the number of collisions
\item Adaptive scheme $I_{n+1}= 2\times I_n$, 
\item $I_0$ fixed, characteristic of the protocol
\end{itemize}

\end{block}
\end{frame}
\begin{frame}{Protocol history}
\begin{block}{University of Hawaiì 1970  \hfill \url{http://www.hicss.hawaii.edu/}}

  \centerline{\includegraphics<1->[width=0.2\textwidth]{images/Norm1.jpg}}%
 Norman Abramson et al.
 
 Use of a radio network to provide computer communications without centralization or vacations
\end{block}
\begin{alertblock}<2->{}
Ancestor of CSMA/CD (ethernet), CSMA/CA (WiFi)...
\end{alertblock}
\end{frame}

\begin{frame}{Quantitative specification validation}
\begin{block}{Experiment}
Propose an experiment to check the specification of the protocol
\end{block}
\begin{alertblock}<2->{Estimation}
How could $I_0$ be estimated ?
\end{alertblock}
\begin{alertblock}<3->{Decision}
How could you conclude on the validity of the implementation of the protocol ?
\end{alertblock}

\end{frame}
%%%%%%%%%%%%%%%%%%%%%%%%%%%%%%%%%%%%%%%%%%%%%%%%%%%%%%
\section[{\scshape Model}]{{\scshape Modelling} : Stochastic Model }

\begin{frame}
\frametitle{Samples}
\textbf{The observations :} a sequence of $n$ waiting times. 

We suppose that the experiment  has been well driven  (no outliers, ...) and the observed values  denoted by
\[
\{x_1,\cdots, x_n\}.
\]
\textbf{The stochastic model :}
the observations are considered to be  realizations of independent random variables with the same probability law $F$
\[
\{X_1,\cdots , X_n\}
\]
Question : \alert{What could be said on $F$ from the observations $\{x_1,\cdots, x_n\}$ ?}

\begin{exampleblock}{A priori knowledge : }
\begin{itemize}
\item The shape of the law is known and depends on some parameter(s) unknown : \alert{\textbf{parametric estimation}}

ex: $X$ follows a uniform distribution on $[0,\theta[$ with $\theta$ unknown
\[
F_\theta(x)=
\begin{cases} 
0 & x\leq 0;\\
\frac 1 \theta x & x\in [0,\theta[;\\
1 & \theta \leq x.
\end{cases}
\]
\item The shape of the law is unknown and some parameters are under study (expectation, variance, moments,...) : \alert{\textbf{non-parametric estimation}}
\end{itemize}
\end{exampleblock}
\end{frame}
\begin{frame}
\frametitle{Basic concepts}
\begin{itemize}
\item A \textbf{statistic} is a function of the observations : $t_n(x_1,\cdots, x_n)$, it usually summarize some parameter of the distribution.
\item An \textbf{estimator} is a random variable $T_n=t(X_1, \cdots, X_n)$ (model of the statistic)
\end{itemize}
\begin{exampleblock}{Example}
\[
t(x_1,\cdots, x_n)=\max_{1\leq i \leq n}x_i \text{ is a statistic on the samples; }
\] 
\[\text{ and  } T_n=\max_{1\leq i \leq n}X_i \text{ the corresponding estimator.}
\]

Law of $T_n$ under the hypothesis $X_i$ uniformly distributed on $[0,\theta[$ :
\[
F_n^\theta(x)=\Prob(\max_{1\leq i \leq n}X_i\leq x) = \frac 1{\theta^n} x^n \text{ using independence and uniformity law of } X_i;
\]
and density
\[
f_n^\theta(x)=\frac 1{\theta^n} n x^{n-1}.
\]

\end{exampleblock}
\end{frame}
\begin{frame}
\frametitle{Bias}
An estimator $T_n$ of some parameter $\theta$ is \textbf{unbiased} iff 
\[
\Esp T_n =\theta
\]
\begin{example}{Bias of estimator $T_n$}

\[
\Esp T_n =\int_0^\theta x f_n^\theta(x) dx=\int_0^\theta x\frac 1{\theta^n} n x^{n-1}dx=\frac 1{\theta^n} n \left[\frac {x^{n+1}}{n+1}\right]_0^\theta=\frac n {n+1}\theta \neq \theta.
\]
$T_n$ is a biased estimator, on average it underestimate $\theta$ on average. But, for large samples the bias decreases to $0$.

\[
\lim_{n\to +\infty}\Esp T_n = \theta \;\; T_n \text{ is \alert{aymptotically unbiased} (or consistent)}
\]
To compensate the bias, 
\[
T^\prime_n = \frac {n+1}n T_n,
\]
which is unbiased.
\end{example}
\end{frame}

\begin{frame}
\frametitle{Risk}
The quality $T_n$ of an estimator is evaluated by the \textbf{risk} function
\[
R(T_n)=\Esp( T_n -\theta)^2= \Esp(T_n-\Esp T_n)^2+(\Esp T_n-\theta)^2=\Var T_n +(\Esp T_n-\theta)^2
\]\begin{itemize}
\item $\Var T_n$ :  the concentration of the distribution
\item $(\Esp T_n-\theta)^2$ : impact of the bias.
\end{itemize}

For an unbiased estimator
\[
R(T_n)=\Var T_n
\]
\end{frame}
\begin{frame}
\begin{exampleblock}{Risk of $T^\prime_n$}
For the example
\begin{eqnarray*}
\Var T_n &= &\Esp (T_n-\Esp T_n)^2 =\Esp T_n^2 -(\Esp T_n)^2 = \int_0^\theta x^2 f_n^\theta(x) dx - \left(\frac n{n+1}\theta\right)^2\\
&=&\left(\frac n{n+2}- \frac {n^2}{(n+1)^2}\right)\theta^2
\end{eqnarray*}
\begin{eqnarray*}
\Var T_n^\prime &= &\Var \frac {n+1}nT_n= \frac {(n+1)^2}{n^2}\Var T_n= \frac {(n+1)^2}{n^2}\left(\frac n{n+2}- \frac {n^2}{(n+1)^2}\right)\theta^2\\
&=& \frac 1{n(n+2)}\theta^2
\end{eqnarray*}

\end{exampleblock}
\end{frame}
\begin{frame}{Another estimator}
Consider 
\[
U_n = \frac 2n \sum_{i=1}^n X_i.
\]
\[
\Esp U_n = \frac 2n \sum_{i=1}^n \Esp X_i= \frac 2n \sum_{i=1}^n \frac \theta 2= \theta.
\]
$U_n$ is an unbiased estimator of $\theta$.
\begin{exampleblock}{Risk of $U_n$}
For the example
\begin{eqnarray*}
\Var U_n &= &\Var \left(\frac 2n \sum_{i=1}^n X_i\right);\\
&=&\left(\frac 2n\right)^2 \sum_{i=1}^n \Var X_i \text{ because of the independence of } X_i;\\
&=& \frac 4 {n^2} n \frac {\theta^2}{12}= \frac {\theta^2}{3n}.
\end{eqnarray*}
The risk associated to $U_n$ is much larger than the risk of $T^\prime_n$ so we will prefer $T^\prime_n$.
\end{exampleblock}
\end{frame}
%%%%%%%%%%%%%%%%%%%%%%%%%%%%%%%%%%%%%%%%%%%%%%%%%%%%%%
\section[{\scshape Design}]{{\scshape Design} : how to build an estimator}
% Likelyhood
% example exponential/Poisson/max
\section[{\scshape Decision}]{{\scshape Decision} : testing}
% null hypothesis
% errors
% confidence intervals

\end{document}
