%%%%%%%%%%%%%%%%%%%%%%%%%%%%%%%%%%%%%%%%%%%%%%%%%%%%%%%%%%%%
%%  This Beamer template was created by Cameron Bracken.
%%  Anyone can freely use or modify it for any purpose
%%  without attribution.
%%
%%  Last Modified: January 9, 2009
%%

%\documentclass[xcolor=x11names,compress,8pt]{beamer}
\documentclass[xcolor=x11names,compress,8pt,
handout
]{beamer}

%%% langue
\usepackage[francais]{babel}

\usepackage[T1]{fontenc}
\usepackage[utf8]{inputenc}

%%% mathématiques
\usepackage{amsmath,amsfonts,amssymb,newlfont,latexsym}

 \def\leq{\leqslant}
 \def\geq{\geqslant}
 
 \def\real{\mathbb{R}}
 \def\Prob{\mathbb{P}}
 
 \def\integer{\mathbb{N}}
 \def\relative{\mathbb{Z}}
 \def\Esp{\mathbb{E}}

%%%%% hyperliens
\usepackage{hyperref,url}
\hypersetup{
dvips,
backref=true, %permet d'ajouter des liens dans...
pagebackref=true,%...les bibliographies
hyperindex=true, %ajoute des liens dans les index.
colorlinks=true, %colorise les liens
breaklinks=true, %permet le retour à la ligne dans les liens trop longs
urlcolor= blue, %couleur des hyperliens
linkcolor= blue, %couleur des liens internes
bookmarks=true, %créé des signets pour Acrobat
bookmarksopen=true} 

%% General document %%%%%%%%%%%%%%%%%%%%%%%%%%%%%%%%%%
\usepackage{graphicx}

\graphicspath{{logo/}{images/}{figures/}{photos/}}

\usepackage{figlatex}%

%\usepackage{tikz}
%\usepackage{pgfplots}

%\usetikzlibrary{decorations.fractals}
%%%%%%%%%%%%%%%%%%%%%%%%%%%%%%%%%%%%%%%%%%%%%%%%%%%%%%


%% Beamer Layout %%%%%%%%%%%%%%%%%%%%%%%%%%%%%%%%%%
\useoutertheme[subsection=false,shadow]{miniframes}
\useinnertheme{default}
%\usefonttheme{serif}
\usepackage{palatino}

\usepackage{xcolor}
\usepackage{colortbl}
\usepackage{soul}
\sethlcolor{green}
\definecolor{gris05}{gray}{0.95}

\setbeamerfont{title like}{shape=\scshape}
\setbeamerfont{frametitle}{shape=\scshape}

\setbeamercolor*{lower separation line head}{bg=DeepSkyBlue4} 
\setbeamercolor*{normal text}{fg=black,bg=white} 
\setbeamercolor*{alerted text}{fg=IndianRed4} 
\setbeamercolor*{example text}{fg=black} 
\setbeamercolor*{structure}{fg=black} 

\setbeamercolor*{palette tertiary}{fg=black,bg=black!10} 
\setbeamercolor*{palette quaternary}{fg=black,bg=black!10} 

\renewcommand{\(}{\begin{columns}}
\renewcommand{\)}{\end{columns}}
\newcommand{\<}[1]{\begin{column}{#1}}
\renewcommand{\>}{\end{column}}
%%%%%%%%%%%%%%%%%%%%%%%%%%%%%%%%%%%%%%%%%%%%%%%%%%
\usefonttheme{progressbar}
\useoutertheme{progressbar}
\useinnertheme{progressbar}


\setbeamertemplate{itemize item}[triangle]  
\setbeamertemplate{enumerate item}[diamond] 

\setbeamertemplate{navigation symbols}{}

%\beamertemplatetransparentcovereddynamic

% \setbeamercovered{transparent}

\title[Data Characterization] % (optional, use only with long paper titles)
{Data Characterization}
\subtitle{on the nature of observations}

\author% (optional, use only with lots of authors)
{		
Lucas Mello Schnorr, Jean-Marc Vincent
}
% - Give the names in the same order as the appear in the paper.
% - Use the \inst{?} command only if the authors have different
%   affiliation.

\institute[LICIA] % (optional, but mostly needed)
{%
{\large INF/UFRGS\\
Porto Alegre, Brazil – October 21th, 2017}
}
% - Use the \inst command only if there are several affiliations.
% - Keep it simple, no one is interested in your street address.

\date[Porto Alegre 2017] % (optional, should be abbreviation of conference name)
{
\includegraphics[width=2cm]{logo/ufrgs2.png}\hfill
\includegraphics[width=2cm]{logo/licia-small.png}\hfill
\includegraphics[width=2cm]{logo/uga.png}
}



% \AtBeginSection[]
%{
%  \begin{frame}<beamer>
%    \frametitle{Représentation de l'information}
%    \tableofcontents[currentsection]
%  \end{frame}
%}

\begin{document}

\begin{frame}
\titlepage
\end{frame}
%%%%%%%%%%%%%%%%%%%%%%%%%%%%%%%%%%%%%%%%%%%%%
\section[{\scshape Data Production}]{{\scshape The Problem} : The Data Set}

\begin{frame}{Data Production}
\begin{alertblock}{First question: Why  this dataset has been produced ? (purpose) }
\begin{itemize}
\item Who organized the study ?
\item What was the question to be answered by the statistical analysis ?
\item Who will be  the target of the  analysis ?
\end{itemize}
\end{alertblock}
\pause
\begin{alertblock}{Second question: Which approach has been used  ? (method) }
\begin{itemize}
\item Exhaustive collected information  
\item Designed survey on a population  
\item Designed Experiments
\end{itemize}
\end{alertblock}
\pause
\begin{alertblock}{Third question: How this dataset has been practically produced ? (observations) }
\begin{itemize}
\item Nature of the items in the Data set  
\item Characterization of data  
\item Semantic of Data
\end{itemize}
\end{alertblock}
\centerline{\colorbox{yellow!85}{\textcolor{black}{\textbf{\large Take time to analyse the production process}}}}
\end{frame}

\section[{\scshape Set of Variables}]{{\scshape Set of Variables} }

\begin{frame}{Analysis of  the Set of  Variables}
\begin{alertblock}{Identification of the variables types}
\begin{itemize}
\item Type of the variables (numbers, identifiers, ...)
\item Set of values taken by the variables (bounds, sets,...)
\item Properties of the variables (positive,...)
\end{itemize}
\end{alertblock}
\pause
\begin{alertblock}{Identification of the variables role}
\begin{itemize}
\item When these variables has been collected ?
\item Why these variables have been chosen ?
\end{itemize}
\end{alertblock}
\pause
\begin{alertblock}{Identification of the variables semantic}
\begin{itemize}
\item What is the interpretation of the variables values ? (size, weight, ...)
\item What are the relations between variables (structure) ?
\end{itemize}
\end{alertblock}
\vfill 

\centerline{\colorbox{yellow!85}{\textcolor{black}{\textbf{\large Take time to build a serious metadata document}}}}

\end{frame}
\section[{\scshape Types}]{{\scshape Types of Variables} }
\begin{frame}{Analysis of  the Type of  Variables}

\begin{alertblock}{Nominal Variables : classification, membership (qualitative)}
\begin{itemize}
\item Values in an unstructured set
\item Examples : color, gender, ...
\item Methods : grouping 
\item Operators : $=$, $\neq$
\end{itemize}
\end{alertblock}
\pause
\begin{alertblock}{Ordinal Variables : Comparison, Level (qualitative)}
\begin{itemize}
\item Values in an ordered set
\item Examples : ranking, opinion, ...
\item Methods :  sorting 
\item Operators : $\leq$, $\geq$
\end{itemize}
\end{alertblock}
\pause
\begin{alertblock}{Quantitative Variables : Quantities}
\begin{itemize}
\item Real values (ratio is significant)
\item Examples : amount, duration, cost ...
\item Methods :  sum, difference 
\item Operators : $+$, $-$, ($\times$, $/$ )
\end{itemize}
\end{alertblock}
\vfill

\centerline{\colorbox{yellow!85}{\textcolor{black}{\textbf{\large Take time to define precisely the variables properties}}}}
\end{frame}

\section[{\scshape Role of Variables}]{{\scshape Role of Variables} }
\begin{frame}{Usage of  Variables}

\begin{alertblock}{Response Variables }
\begin{itemize}
\item Quantity   asked by the  question 
\item Examples : response time, iteration duration, ...
\end{itemize}
\end{alertblock}
\pause
\begin{alertblock}{Explanatory Variables}
\begin{itemize}
\item Variables that could affect the response variable
\item Examples : size, load, ...
\end{itemize}
\end{alertblock}
\pause
\begin{alertblock}{Univariate or Multivariate}
\begin{itemize}
\item Univariate : one variable is involved
\item Multivariate : several variables are involved
\end{itemize}
\end{alertblock}
\vfill

\centerline{\colorbox{yellow!85}{\textcolor{black}{\textbf{\large Take time to identify the response/explanatory variables }}}}
\end{frame}

\end{document}

%%%%%%%%%%%%%%%%%%%%%%%%%%%%%%%%%%%%%%%%%%%%%%%%%%%%%%
%%%%%%%%%%%%%%%%%%%%%%%%%%%%%%%%%%%%%%%%%%%%%%%%%%%%%%

